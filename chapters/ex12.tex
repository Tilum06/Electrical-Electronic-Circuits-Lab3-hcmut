\section{BJT’s logic gate application}

Figure~\ref{fig:ex12_question} describes a straightforward NOT gate theoretical implementation using an NPN bipolar junction transistor. In the circuit, the NPN junction transistor operates in the saturation mode.

\begin{figure}[H]
    \centering
    \includegraphics[width=0.7\textwidth]{graphics/ex12_question.png}
    \caption{NPN theoretical NOT gate}
    \label{fig:ex12_question}
\end{figure}

\noindent V1 = 0 When the source is off, the voltage would be 0V.\\
V2 = 5 When the source is on, the voltage would be 5V.\\
TD = 0 Delay time. This exercise assumes that there is no delay.\\
TR = 5ns The rise time of the pulse (from off to on stage).\\
TF = 3ns The fall time of the pulse (from on to off stage).\\
PW = 50ms Pulse width: The time in which the source keeps on.\\
PER = 100ms The period of the signal.\\

\subsection{Simulation}

\textbf{Your images goes here}

The red line is the input pulse signal, while the green line is the output pulse signal.

\begin{figure}[H]
    \centering
    \includegraphics[width=0.9\textwidth]{graphics/ex12_sim_a.png}
    \caption{Simulation result of NPN theoretical NOT gate}
    \label{fig:ex12_simulation}
\end{figure}