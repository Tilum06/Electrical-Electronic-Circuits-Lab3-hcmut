\section{Opto}
The element OK1 in Figure 1.18 is an optocoupler, which includes a light-emitting diode
(LED) and a photodiode. The photodiode's conductivity depends on the intensity of the
light emitted by the LED and, of course, on the current through the LED. When the voltage
across the LED is lower than its barrier potential, the optocoupler is cut off. When there
is current through the LED, the optocoupler is in the transfer mode. Like the current
gain $\beta$ of a BJT, the optocoupler also has a current transfer ratio (CTR). Assume
the LED has a barrier potential $V_F = 1.7\text{V}$ and the optocoupler has CTR = 2.
Calculate the voltage $V_{OUT}$ when the switch is closed. Finally, give your idea about
what we may use an optocoupler for, and how to use it?
\begin{figure}[H]
  \centering
  \includegraphics[width=0.8\textwidth]{graphics/ex13-circuit.png}
  \caption{Optocoupler circuit}
  \label{fig:ex13-circuit}
\end{figure}

\subsection{Solution}
\begin{align*}
    I_F = I_{R1} = \frac{V_{BAT} - V_F}{R_1} = \frac{9V - 1.7V}{10k\Omega} = 0.73 mA \\
    I_{R2} = I_C = CTR * I_F = 2 * 0.73 mA = 1.46 mA \\
    V{OUT} = 5V - (I_C * R_2) = 5V - (1.46 mA * 4.7k\Omega) = -1.86 V \\
\end{align*}

We see that the output voltage $V_{OUT}$ is negative, which means that the transistor is in saturation mode. Therefore, we can approximate:
\begin{align*}
    V_{OUT} \approx V_{CE(sat)} \approx 0.2 V \\
\end{align*}

When the switch is opened, there is no current through the LED, and the optocoupler is cut off. Therefore, there is no current through resistor R2, and we have:
\begin{align*}
    V_{OUT} = 5V \\
\end{align*}

The opto sensor can be used as a non-contact switch or position / speed sensor. It consists of an infrared LED and a phototransistor facing each other. When the light path is not blocked, the phototransistor conducts; when an object interrupts the beam, the collector current decreases and the output voltage changes.
In practice, we bias the LED with a current-limiting resistor and connect the phototransistor in a pull-up configuration. By placing a rotating disc or moving object inside the slot, the sensor generates pulses at its output, which can be counted or measured by an oscilloscope or a microcontroller to determine speed or position.