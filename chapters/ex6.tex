\section{PNP Circuit}

Figure~\ref{fig:ex6-question} shows a very typical PNP transistor circuit. Calculate $I_B$ , $I_E$ , and $I_C$ then simulate the circuit to double-check your calculation. Assume the current gain $\beta = 100$.

\begin{figure}[H]
    \centering
    \includegraphics[width=0.7\textwidth]{graphics/ex6_question.png}
    \caption{A PNP Circuit.}
    \label{fig:ex6-question}
\end{figure}

\subsection{Theoretical Calculation}

\textbf{\textit{Notes:}}

\textit{Explanations, formulas, and equations are expected rather than only results.}\\

Because the base–emitter junction of a PNP transistor is forward-biased, the emitter is approximately 0.7 V higher than the base. Therefore,
\[V_{EB} = 0.7~\text{V}\]

According to KVL at the base-emitter loop,
\[I_B = \frac{V_{EE} - V_{EB} - V_{BB}}{R_B} = \frac{12~\text{V} - 0.7~\text{V} - 8~\text{V}}{40\text{k}\Omega} = 82.5~\mu\text{A}\]

Calculate the collector current $I_C$:
\[I_C = \beta \cdot I_B = 100 \cdot 82.5~\mu\text{A} = 8.25~\text{mA}\]

Calculate the emitter current $I_E$:
\[I_E = I_C + I_B = 8.25~\text{mA} + 0.0825~\text{mA} = 8.3325~\text{mA}\]

\subsection{Simulation}

\textbf{Your image goes here:}
\begin{figure}[H]
    \centering
    \includegraphics[width=0.7\textwidth]{graphics/ex6_sim.png}
    \caption{Simulation results for the PNP transistor circuit.}
    \label{fig:ex6-sim}
\end{figure}

\subsection{Compare}

\[
I_B \text{(in theory)} = 82.5~\mu\text{A}
\qquad
I_B \text{(simulation)} = 79.31~\mu\text{A}
\]

\[
I_C \text{(in theory)} = 8.25~\text{mA}
\qquad
I_C \text{(simulation)} = 7.931~\text{mA}
\]

\[
I_E \text{(in theory)} = 8.3325~\text{mA}
\qquad
I_E \text{(simulation)} = 8.010~\text{mA}
\]

