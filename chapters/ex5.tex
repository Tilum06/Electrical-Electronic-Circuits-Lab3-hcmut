\section{Exercise 5: Simple bias configuration}
The circuit given in \ref{fig:ex5-circuit} is known as a simple kind of NPN bias configuration. First,
students simulate the circuit with two values of RC, respectively 10 Ohms and 1k Ohms.
Then, give your statement on the change of the current $I_E$ and explain the phenomena.
\begin{figure}[H]
  \centering
  \includegraphics[width=0.8\textwidth]{graphics/ex5-circuit.png}
  \caption{Simple bias configuration circuit}
  \label{fig:ex5-circuit}
\end{figure}

\subsection{Simulation}
With $R_C$ = 10 Ohms, the simulation result is shown in Figure \ref{fig:ex5-simulation-10ohm}.
\begin{figure}[H]
  \centering
  \includegraphics[width=0.8\textwidth]{graphics/ex5-simulation-10ohm.png}
  \caption{Simulation result with $R_C$ = 10 Ohms}
  \label{fig:ex5-simulation-10ohm}
\end{figure}


With $R_C$ = 1k Ohms, the simulation result is shown in Figure \ref{fig:ex5-simulation-1kohm}.
\begin{figure}[H]
  \centering
  \includegraphics[width=0.8\textwidth]{graphics/ex5-simulation-1kohm.png}
  \caption{Simulation result with $R_C$ = 1k Ohms}
  \label{fig:ex5-simulation-1kohm}
\end{figure}

\subsection{Circuit analysis}
We will redraw the circuit as in Figure \ref{fig:ex5-thevenin} for easier analysis, using Thevenin's theorem.
\begin{figure}[H]
  \centering
  \includegraphics[width=0.6\textwidth]{graphics/ex5-thevenin.png}
  \caption{Thevenin equivalent circuit}
  \label{fig:ex5-thevenin}
\end{figure}
Calculating Thevenin's equivalent voltage and resistance:
\begin{align*}
    V_{TH} &= V_{CC} * \frac{R_2}{R_1 + R_2} = 12V * \frac{10k\Omega}{80k\Omega + 40k\Omega} = 4V \\
    R_{TH} &= R_1 || R_2 = \frac{R_1 * R_2}{R_1 + R_2} = \frac{80k\Omega * 40k\Omega}{80k\Omega + 40k\Omega} = 26.67k\Omega
\end{align*}

Applying KVL in the input loop, we have:
\begin{align*}
    V_{TH} &= I_B * R_{TH} + I_E * R_E + V_{BE} \\
    I_E &= (1 + \beta) * I_B \\
    V_{TH} &= I_B * R_{TH} + (1 + \beta) * I_B * R_E + V_{BE} \\
    I_B &= \frac{V_{TH} - V_{BE}}{R_{TH} + (1 + \beta) * R_E} = \frac{4 - 0.7}{26.67k\Omega + (1 + 100) * 3.3k\Omega} = 9.17 \mu A \\
    I_E &= I_B * (1 + \beta) = 9.17 \mu A * 101 = 0.927 mA \\
    I_C &= \beta * I_B = 9.17 \mu A * 100 = 0.917 mA
\end{align*}
\clearpage
\textbf{When $R_C$ = 10 Ohms:} 

The voltage drop across $R_C$ is $V_{RC} = I_C * R_C = 0.917 mA * 10 \Omega = 9.17 mV$. 

This voltage drop is very small compared to $V_{CC}$, so the collector voltage $V_C$ remains high, allowing the transistor to operate in the active region. 

Therefore, the emitter current $I_E$ is approximately 0.927 mA.

\textbf{When $R_C$ = 1k Ohms:} 

The voltage drop across $R_C$ is $V_{RC} = I_C * R_C = 0.917 mA * 1k \Omega = 0.917 V$. 

This voltage drop is significant compared to $V_{CC}$, which reduces the collector voltage $V_C$. If $V_C$ drops below the base voltage minus $V_{BE}$, the transistor will enter saturation, causing a further increase in $I_E$. 

However, in this case, the calculated $I_E$ remains approximately 0.927 mA, but the actual current may be limited by the supply voltage and the saturation of the transistor.
